%%%%%%%%%%%%%%
%% PREAMBLE %%
%%%%%%%%%%%%%%

\documentclass[11pt, letterpaper, sans]{moderncv}
\moderncvstyle{classic}
\moderncvcolor{grey}
\renewcommand{\familydefault}{\sfdefault}
\usepackage{tikz}
\usepackage{xcolor}
\usepackage{multicol}
\usepackage{fontawesome}
\usepackage[utf8]{inputenc}
\usepackage{hanging}
\usepackage{geometry}
\geometry{left=1.5cm, right=1.5cm, top=0.5cm, bottom=1.5cm, heightrounded}

%%%%%%%%%%%%%%%%%%%
%% PERSONAL DATA %%
%%%%%%%%%%%%%%%%%%%

\firstname{Denis} \familyname{Jacob~Machado,~Ph.D.}
\address{}{}{Charlotte--NC, USA}
\homepage{phyloinformatics.com}
\email{dmachado@uncc.edu}
\phone{+1 (704) 687-8564}

%%%%%%%%%%%%%%%%%%%%%%%
%% HEADER AND FOOTER %%
%%%%%%%%%%%%%%%%%%%%%%%

\usepackage{xcolor}
\usepackage{etoolbox}
\usepackage{fancyhdr}
\definecolor{headercolor}{rgb}{0.35,0.35,0.35} % dark grey
\pagestyle{fancy}
\fancypagestyle{plain}{\fancyhf{}}
\renewcommand{\headrulewidth}{0pt}
\headheight 15pt
\footskip 45pt
\renewcommand{\footrulewidth}{0pt}
\newcommand{\helv}{
\fontsize{8}{10}\selectfont}
\makeatletter
\patchcmd{\@fancyhead}{\rlap}{\color{headercolor}\rlap}{}{}
\patchcmd{\headrule}{\hrule}{\color{headercolor}\hrule}{}{}
\patchcmd{\@fancyfoot}{\rlap}{\color{headercolor}\rlap}{}{}
\patchcmd{\footrule}{\hrule}{\color{headercolor}\hrule}{}{}
\makeatother
\fancyfoot[L]{\color{headercolor}\nouppercase{\helv\small{}}}
\fancyfoot[R]{\color{headercolor}\nouppercase{\helv\small{}}}
\nopagenumbers{}

\begin{document}\thispagestyle{empty}\maketitle

%%%%%%%%%%%%%%%%
%% HIGHLIGHTS %%
%%%%%%%%%%%%%%%%

\vspace{-3em}

\textbullet~ \textbf{Bioinformaticist} with strong biology, biostatistics, and programming background.

\textbullet~ Peer-reviewed publications on topics including \textbf{genomics}, \textbf{evolution}, \textbf{parasitology}, \textbf{regeneration}, \textbf{emerging infectious diseases}, and \textbf{genomic epidemiology}.

\textbullet~ $3+$ years of experience using \textbf{machine learning} (e.g., PyTorch, TensorFlow), \textbf{predictive modeling}, \textbf{high-throughput data processing}, and \textbf{data mining} algorithms to solve practical problems.

\textbullet~ $10+$ years of experience in \textbf{molecular biology}, \textbf{high-throughput sequencing} (e.g., Illumina, PacBio, Hi-C, GWAS, and RNA-Seq), and sequence assembly, and differential gene expression analyses.

% \textbullet~ Concepts used in my current and past projects include Python, Flask, R, Shiny, Tableu, SQL, Git, Jupyter, AlphaFold2, RoseTTAFold, HADDOCK, SQL, and AWS.

%%%%%%%%%%%%%%%%%%%%
%% POSITIONS HELD %%
%%%%%%%%%%%%%%%%%%%%

\vspace{-0.5em}

\section{Work Experience}
\cventry{Aug/15/22--Current}
{Assistant Professor in Bioinformatics}
{University of North Carolina at Charlotte}
{Charlotte--NC}
{USA}
{
\textbullet~First line of research: Investigate new pathogens’ emergence, evolution, and spread, focusing on preventing and treating infectious diseases.\\
\textbullet~Second line of research: Create computational and molecular solutions to make data from biorepositories more readily available to biomedical research.\\
\textbullet~For more information, please visit \href{https://phyloinformatics.com/}{phyloinformatics.com}.\\
}
%
\cventry{2019--2022}
{Postdoctoral Researcher}
{University of North Carolina at Charlotte}
{Charlotte--NC}
{USA}
{
\textbullet~I studied the origins, evolution, and zoonotic events of coronaviruses. I also employed big data analysis to categorize different variants of SARS-CoV-2 and applied deep learning techniques to produce highly accurate structure predictions of their proteins.\\
\textbullet~I predict how structural changes in the receptor-binding domain pf the spike protein of different variants of SARS-CoV-2 may reduce antibody interaction without completely evading existing neutralizing antibodies.\\ % (and therefore current vaccines).\\
\textbullet~I created the first programs for gene prediction and annotation of \textit{Orthocoronavirinae} (including SARS-COV-2) and \textit{Flaviviridae} (including Hepatites C, yellow fever, dengue, and Zika virus).\\ % and deployed the pipelines as a web application.\\
\textbullet~Produced draft genomes of highly regenerative echinoderms such as brittle stars and sea cucumbers for comparative genomics and tissue regeneration research.
%\textbullet~My current research focus on evolutionary approaches to transform multi-omic data into knowledge and accelerate the pace of antiviral drug discovery and development.
}
%
\cventry{2018--2019}
{Research Collaborator}
{University of São Paulo}
{São Paulo--SP}
{Brazil}
{
\textbullet~I built and administered the first computer clusters at the Muzeum of Zoology of the University of São Paulo (MUZUSP).\\ % , providing training and technology to bridge the gap between basic animal research and applied bioinformatics with potential biomedical significance.\\
\textbullet~I participated in the planning, funding acquisition, and implementation of the first laboratory to sequence historical DNA at the Department of Zoology of the University of São Paulo (USP).\\ % (e.g., from degraded museum samples)
\textbullet~I developed new genome skimming techniques to retrieve genomic data of non-model organisms when samples were small or degraded. I also created the necessary software to facilitate the analysis of organelle genomes.\\ % , including new indices that check for the completeness of circular genomes or misalignments in other types of contigs and scaffolds.
}
%
\cventry{2018--2019}
{Invited Lecturer}
{University of São Paulo}
{São Paulo--SP}
{Brazil}
{
\textbullet~I created and taught the first graduate-level courses in bioinformatics and computer programming for biologists at USP's Graduate Program in Zoology. % , which is ranked first in animal research worldwide.
}
%
\cventry{Jul--Aug 2017}
{Invited Lecturer}
{University of Magdalena}
{Santa Marta D.T.C.H.}
{Colombia}
{
\textbullet~I designed and taught the first course in bioinformatics at the University of Magdalena. Classes included practical and theoretical lessons on DNA extraction and isolation, high-throughput sequencing, sequence alignment, reference, and \textit{de novo} genome assembly, and gene prediction and annotation.\\
\textbullet~I also offered consultation to faculty members involved in animal research who needed web and dry lab solutions for next-generation sequence analyses.
}

%%%%%%%%%%%%%%%%%%%%%%
%% FORMAL EDUCATION %%
%%%%%%%%%%%%%%%%%%%%%%

\vspace{-1em}

\section{Education}

\cventry{2013--2018}
{Ph.D. in Bioinformatics}
{University of São Paulo (USP)}
{São Paulo--SP}
{Brazil}
{}

\cventry{2010--2012}
{Master's in Zoology (Parasitology)}
{University of São Paulo (USP)}
{São Paulo--SP}
{Brazil}
{}

\cventry{2006--2009}
{Bachelor's in Biological Sciences (Marine Biology)}
{São Paulo State University (UNESP)}
{São Vicente--SP}
{Brazil}
{}

You will find my \textit{curriculum vitae} and other professional information at \href{https://phyloinformatics.com/members/Denis_Jacob_Machado.html}{phyloinformatics.com}.

\end{document}
