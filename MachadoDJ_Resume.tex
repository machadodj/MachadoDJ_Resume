%%%%%%%%%%%%%%
%% PREAMBLE %%
%%%%%%%%%%%%%%

\documentclass[11pt, letterpaper, sans]{moderncv}

\moderncvstyle{classic}
\moderncvcolor{grey}
\renewcommand{\familydefault}{\sfdefault}

\usepackage{tikz}
\usepackage{xcolor}
\usepackage{multicol}
\usepackage[hmargin=1in,vmargin=1in]{geometry}
\usepackage[english]{babel}
\usepackage{xcolor}
\usepackage{etoolbox}
\usepackage{fancyhdr}
\usepackage{hanging}

\ifxetexorluatex
  \usepackage{fontspec}
  \usepackage{unicode-math}
  \defaultfontfeatures{Ligatures=TeX}
  \setmainfont{Latin Modern Roman}
  \setsansfont{Latin Modern Sans}
  \setmonofont{Latin Modern Mono}
  \setmathfont{Latin Modern Math}
\else
  \usepackage[utf8]{inputenc}
  \usepackage[T1]{fontenc}
  \usepackage{lmodern}
\fi

%%%%%%%%%%%%%%%%%%%
%% PERSONAL DATA %%
%%%%%%%%%%%%%%%%%%%

\firstname{Denis} \familyname{Jacob~Machado}
    \address{Charotte--NC, USA}{}{}
    \homepage{phyloinformatics.com}
    \email{dmachado@uncc.edu}
    \phone{+1 (704) 687-8564}

\begin{document}

\maketitle

%%%%%%%%%%%%%%%%
%% HIGHLIGHTS %%
%%%%%%%%%%%%%%%%

\begin{itemize}
    \item \textbf{Computational biologist} integrating molecular biology, bioinformatics, and evolutionary genomics to study health-relevant organisms and emerging infectious diseases.
    \item Expertise in \textbf{genomics}, \textbf{phylogenetics}, and \textbf{machine learning} for large-scale biological data analysis, including \textbf{genomic epidemiology} and \textbf{AI-driven discovery}.
    \item $15+$ years of experience in \textbf{molecular biology}, \textbf{high-throughput sequencing} (Illumina, PacBio, RNA-Seq, Hi-C), genome assembly, and transcriptomic analysis.
    \item $7+$ years applying \textbf{predictive modeling}, \textbf{data mining}, and \textbf{AI/ML frameworks} (PyTorch, TensorFlow) to biological and public-health problems.
    \item Proven record of \textbf{peer-reviewed publications}, leadership in \textbf{computational phylogenetics}, and mentoring of multidisciplinary research teams.
\end{itemize}

%%%%%%%%%%%%%%%%%%%%
%% POSITIONS HELD %%
%%%%%%%%%%%%%%%%%%%%

\section{Positions Held}

\cventry{2022--Current}
    {Assistant Professor in Bioinformatics}
    {University of North Carolina at Charlotte}
    {Charlotte--NC}
    {USA}
    {}
    % \begin{itemize}
    %     \item Investigate new pathogens' emergence, evolution, and spread focusing on preventing and treating infectious diseases.\\
    %     \item Create computational and molecular solutions to make data from biorepositories more readily available to biomedical research.
    %     \item First line of research: Investigate new pathogens’ emergence, evolution, and spread, focusing on preventing and treating infectious diseases.
    %     \item Second line of research: Create computational and molecular solutions to make data from biorepositories more readily available to biomedical research.
    %     \item For more information, please visit \href{https://phyloinformatics.com/}{phyloinformatics.com}.
    % \end{itemize}

\cventry{2019--2022}
    {Postdoctoral Researcher}
    {University of North Carolina at Charlotte}
    {Charlotte--NC}
    {USA}
    {}
    % \begin{itemize}
    %     \item I conducted the most diverse and complete analysts of the evolution of coronaviruses (e.g., SARS-CoV, MERS-CoV, and SARS-CoV-2) and flaviviruses (e.g., dengue, zika, and west-nile virus)
    %     \item Produced draft genomes of highly regenerative echinoderms such as brittle stars and sea cucumbers for
    %     comparative genomics and tissue regeneration research
    %     \item I studied the origins, evolution, and zoonotic events of coronaviruses. I also employed big data analysis to categorize different variants of SARS-CoV-2 and applied deep learning techniques to produce highly accurate structure predictions of their proteins.
    %     \item I predict how structural changes in the receptor-binding domain pf the spike protein of different variants of SARS-CoV-2 may reduce antibody interaction without completely evading existing neutralizing antibodies.
    %     \item I created the first programs for gene prediction and annotation of \textit{Orthocoronavirinae} (including SARS-COV-2) and \textit{Flaviviridae} (including Hepatites C, yellow fever, dengue, and Zika virus).
    %     \item Produced draft genomes of highly regenerative echinoderms such as brittle stars and sea cucumbers for comparative genomics and tissue regeneration research.
    %     \item My current research focus on evolutionary approaches to transform multi-omic data into knowledge and accelerate the pace of antiviral drug discovery and development.
    % \end{itemize}

\cventry{2018--2019}
    {Research Collaborator}
    {University of São Paulo}
    {São Paulo--SP}
    {Brazil}
    {}
    % \begin{itemize}
    %     \item I built and administered the first computer clusters at the Muzeum of Zoology of the University of São Paulo (MUZUSP).
    %     \item I participated in the planning, funding acquisition, and implementation of the first laboratory to sequence historical DNA at the Department of Zoology of the University of São Paulo (USP).
    %     \item I developed new genome skimming techniques to retrieve genomic data of non-model organisms when samples were small or degraded. I also created the necessary software to facilitate the analysis of organelle genomes.
    % \end{itemize}
%
\cventry{2018--2019}
    {Invited Lecturer}
    {University of São Paulo}
    {São Paulo--SP}
    {Brazil}
    {}
    % \begin{itemize}
    %     \item I created and taught the first graduate-level courses in bioinformatics and computer programming for biologists at USP's Graduate Program in Zoology.
    % \end{itemize}
%
\cventry{2017, 2025}
    {Invited Lecturer}
    {University of Magdalena}
    {Santa Marta D.T.C.H.}
    {Colombia}
    {
    % \begin{itemize}
    %     \item I designed and taught the first course in bioinformatics at the University of Magdalena. Classes included practical and theoretical lessons on DNA extraction and isolation, high-throughput sequencing, sequence alignment, reference, and \textit{de novo} genome assembly, and gene prediction and annotation.
    %     \item I also offered consultation to faculty members involved in animal research who needed web and dry lab solutions for next-generation sequence analyses.
    % \end{itemize}
    }

%%%%%%%%%%%%%%%%%%%%%%
%% FORMAL EDUCATION %%
%%%%%%%%%%%%%%%%%%%%%%

\section{Education}

\cventry{2013--2018}
    {Ph.D. in Bioinformatics}
    {University of São Paulo (USP)}
    {São Paulo--SP}
    {Brazil}
    {}

\cventry{2010--2012}
    {Master's in Zoology (Parasitology)}
    {University of São Paulo (USP)}
    {São Paulo--SP}
    {Brazil}
    {}

\cventry{2006--2009}
    {Bachelor's in Biological Sciences (Marine Biology)}
    {São Paulo State University (UNESP)}
    {São Vicente--SP}
    {Brazil}
    {}

%%%%%%%%%%%%%%%%%%
%% PUBLICATIONS %%
%%%%%%%%%%%%%%%%%%

\section{Selected Publications}

    \begin{hangparas}{.2in}{1}
    
    Márquez, R., \textbf{Jacob Machado, D.}, Nouri, R., Gendreau, K. L., Janies, D., Saporito, R. A., Kronforst, M. R., \& Grant, T. (2025). A draft genome assembly for the dart-poison frog \textit{Phyllobates terribilis}. \textit{Gigabyte}. \href{https://doi.org/10.46471/gigabyte.157}{10.46471/gigabyte.157}.
    
    Alves, P. V., da Silva, R. J., Scholz, T., de Chambrier, A., Luque, J. L., Duchenko, A., Janies, D., \& \textbf{Jacob Machado, D.} (2025). Machine learning models accurately predict clades of proteocephalidean tapeworms (Onchoproteocephalidea) based on host and biogeographical data. \textit{Cladistics}. \href{https://doi.org/10.1111/cla.12610}{10.1111/cla.12610}.
    
    Nouri, R., Mashanov, V., Harris, A., New, G., Taylor, W., Janies, D., Reid, R. W., \& \textbf{Jacob Machado, D.} (2024). Unveiling putative modulators of mutable collagenous tissue in the brittle star \textit{Ophiomastix wendtii}: an RNA-Seq analysis. \textit{BMC Genomics}. \href{https://doi.org/10.1186/s12864-024-10926-7}{10.1186/s12864-024-10926-7}.
    
    \textbf{Jacob Machado, D.}, Marques, F. P. L., Jiménez-Ferbans, L., \& Grant, T. (2022). An empirical test of the relationship between the bootstrap and likelihood ratio support in maximum likelihood phylogenetic analysis. \textit{Cladistics}. \href{https://doi.org/10.1111/cla.12496}{10.1111/cla.12496}.
    
    \textbf{Jacob Machado, D.}, Scott, R., Guirales, S., \& Janies, D. A. (2021). Fundamental evolution of all Orthocoronavirinae including three deadly lineages descendent from Chiroptera-hosted coronaviruses: SARS-CoV, MERS-CoV and SARS-CoV-2. \textit{Cladistics}. \href{https://doi.org/10.1111/cla.12454}{10.1111/cla.12454}.
    
    Grant, T., Rada, M., Anganoy-Criollo, M., Batista, A., Dias, P. H., Jeckel, A. M., \textbf{Jacob Machado, D.}, \& Rueda-Almonacid, J. V. (2017). Phylogenetic systematics of dart-poison frogs and their relatives revisited (Anura: Dendrobatoidea). \textit{South American Journal of Herpetology}, 12(s1), S1--S90. \href{https://doi.org/10.2994/sajh-d-17-00017.1}{10.2994/sajh-d-17-00017.1}.
    
    Trevisan, B., Alcantara, D. M. C., \textbf{Machado, D. J.}, Marques, F. P. L., \& Lahr, D. J. G. (2019). Genome skimming is a low-cost and robust strategy to assemble complete mitochondrial genomes from ethanol-preserved specimens in biodiversity studies. \textit{PeerJ}, 7, e7543. \href{https://doi.org/10.7717/peerj.7543}{10.7717/peerj.7543}.
    
    Orrico, V. G. D., Grant, T., Faivovich, J., Rivera‐Correa, M., Rada, M. A., Lyra, M. L., Cassini, C. S., Valdujo, P. H., Schargel, W. E., \textbf{Machado, D. J.}, Wheeler, W. C., Barrio‐Amorós, C., Loebmann, D., Moravec, J., Zina, J., Solé, M., Sturaro, M. J., Peloso, P. L. V., Suarez, P., \& Haddad, C. F. B. (2021). The phylogeny of Dendropsophini (Anura: Hylidae: Hylinae). \textit{Cladistics}. \href{https://doi.org/10.1111/cla.12429}{10.1111/cla.12429}.
    
    Ford, C. T., \textbf{Jacob Machado, D.}, \& Janies, D. A. (2022). Predictions of the SARS-CoV-2 Omicron variant (B.1.1.529) spike protein receptor-binding domain structure and neutralizing antibody interactions. \textit{Frontiers in Virology}, 2. \href{https://doi.org/10.3389/fviro.2022.830202}{10.3389/fviro.2022.830202}.
    
    \textbf{Machado, D. J.} (2015). YBYRÁ facilitates comparison of large phylogenetic trees. \textit{BMC Bioinformatics}. \href{https://doi.org/10.1186/s12859-015-0642-9}{10.1186/s12859-015-0642-9}.
    
    \end{hangparas}

%%%%%%%%%%%
%% LINKS %%
%%%%%%%%%%%

\section{Links}

    \begin{itemize}
        \item \textbf{ORCiD:} \url{https://orcid.org/my-orcid?orcid=0000-0001-9858-4515}
        \item \textbf{Google Scholar:} \url{https://scholar.google.com/citations?user=TK6Ms80AAAAJ&hl=pt-BR}
        \item \textbf{LinkedIn:} \url{https://www.linkedin.com/in/machadodj}
        \item \textbf{Phyloinformatics Lab:} \url{https://phyloinformatics.com}
        \item \textbf{UNC Charlotte's CCI:} \url{https://cci.charlotte.edu/directory/denis-jacob-machado}
        \item \textbf{Web of Science:} \url{https://www.webofscience.com/wos/author/record/I-1452-2015}
        \item \textbf{Loop:} \url{https://loop.frontiersin.org/people/1352271/overview}
        \item \textbf{SciProfiles:} \url{https://sciprofiles.com/profile/machadodj}
        \item \textbf{GitHub:} \url{https://github.com/machadodj}
        \item \textbf{GitLab:} \url{https://gitlab.com/MachadoDJc}
    \end{itemize}

\end{document}
