%%%%%%%%%%%%%%
%% PREAMBLE %%
%%%%%%%%%%%%%%

\documentclass[11pt, letterpaper, sans]{moderncv}
\moderncvstyle{classic}
\moderncvcolor{grey}
\renewcommand{\familydefault}{\sfdefault}
\usepackage{tikz}
\usepackage{xcolor}
\usepackage{multicol}
\usepackage{fontawesome}
\usepackage[utf8]{inputenc}
\usepackage{hanging}
\usepackage{geometry}
\geometry{left=1.5cm, right=1.5cm, top=0.5cm, bottom=1.5cm, heightrounded}

%%%%%%%%%%%%%%%%%%%
%% PERSONAL DATA %%
%%%%%%%%%%%%%%%%%%%

\firstname{Denis} \familyname{Jacob~Machado,~Ph.D.}
\address{}{}{Charlotte--NC, EUA}
\email{dmachado@uncc.edu}
\homepage{https://machadodj.github.io}
\mobile{+1 (980) 287-9452}

%%%%%%%%%%%%%%%%%%%%%%%
%% HEADER AND FOOTER %%
%%%%%%%%%%%%%%%%%%%%%%%

\usepackage{xcolor}
\usepackage{etoolbox}
\usepackage{fancyhdr}
\definecolor{headercolor}{rgb}{0.35,0.35,0.35} % dark grey
\pagestyle{fancy}
\fancypagestyle{plain}{\fancyhf{}}
\renewcommand{\headrulewidth}{0pt}
\headheight 15pt
\footskip 45pt
\renewcommand{\footrulewidth}{0pt}
\newcommand{\helv}{
\fontsize{8}{10}\selectfont}
\makeatletter
\patchcmd{\@fancyhead}{\rlap}{\color{headercolor}\rlap}{}{}
\patchcmd{\headrule}{\hrule}{\color{headercolor}\hrule}{}{}
\patchcmd{\@fancyfoot}{\rlap}{\color{headercolor}\rlap}{}{}
\patchcmd{\footrule}{\hrule}{\color{headercolor}\hrule}{}{}
\makeatother
\fancyfoot[L]{\color{headercolor}\nouppercase{\helv\small{}}}
\fancyfoot[R]{\color{headercolor}\nouppercase{\helv\small{}}}
\nopagenumbers{}

\begin{document}\thispagestyle{empty}\maketitle

%%%%%%%%%%%%%%%%
%% HIGHLIGHTS %%
%%%%%%%%%%%%%%%%

\vspace{-3em}

\begin{itemize}

\item \textbf{Bioinformata} com treinamento avançado em biologia, bioestatística, e programação.

\item Artigos publicados em jornais científicos em áreas incluindo \textbf{genômica}, \textbf{evolução}, \textbf{parasitologia}, \textbf{regerenação}, \textbf{doenças infecciosas emergentes} e \textbf{epidemiologia}.

\item $3+$ anos de experiênia em \textbf{aprendizagem de máquina} (e.g., PyTorch, TensorFlow), \textbf{modelagem preditiva}, \textbf{processamento de dados em larga escala (``big data'')} e \textbf{mineiração de dados} para resolver problemas práticos.

\item $10+$ anos de experiência em \textbf{biologia molecular}, \textbf{sequenciamento de alta-performance} (e.g., Illumina, PacBio, Hi-C, GWAS, and RNA-Seq), montagem de genomas e análise de expressão diferencial de genes.

\item Conceitos frequentemente aplicados: Python, Flask, R, Shiny, Tableu, SQL, Git, Jupyter, AlphaFold2, RoseTTAFold, HADDOCK, SQL e AWS.

\end{itemize}

%%%%%%%%%%%%%%%%%%%%
%% POSITIONS HELD %%
%%%%%%%%%%%%%%%%%%%%

\vspace{-0.5em}

\section{Experiência Profissional}
\cventry{2019--atual}
{Pesquisador Pós-doutor}
{Universidade da Carolina do Norte em Charlotte}
{Charlotte--NC}
{EUA}
{
\textbullet~I studied the origins, evolution, and zoonotic events of coronaviruses. I also employed big data analysis to categorize different variants of SARS-CoV-2 and applied deep learning techniques to produce highly accurate structure predictions of their proteins.\\
\textbullet~I predict how structural changes in the receptor-binding domain pf the spike protein of different variants of SARS-CoV-2 may reduce antibody interaction without completely evading existing neutralizing antibodies (and therefore current vaccines).\\
\textbullet~I created the first programs for gene prediction and annotation of \textit{Orthocoronavirinae} (including SARS-COV-2) and \textit{Flaviviridae} (including Hepatites C, yellow fever, dengue, and Zika virus) and deployed the pipelines as a web application.\\
\textbullet~Produced draft genomes of highly regenerative echinoderms such as brittle stars and sea cucumbers for comparative genomics and tissue regeneration research.
%\textbullet~My current research focus on evolutionary approaches to transform multi-omic data into knowledge and accelerate the pace of antiviral drug discovery and development.
}
%
\cventry{2018--2019}
{Pesquisador Colaborador}
{Universidade de São Paulo}
{São Paulo--SP}
{Brasil}
{
\textbullet~I built and administered the first computer clusters at the Muzeum of Zoology of the University of São Paulo (MUZUSP), providing training and technology to bridge the gap between basic animal research and applied bioinformatics with potential biomedical significance.\\
\textbullet~I participated in the planning, funding acquisition, and implementation of the first laboratory to sequence historical DNA (e.g., from degraded museum samples) at the Department of Zoology of the University of São Paulo (USP).\\
\textbullet~I developed new genome skimming techniques to retrieve genomic data of non-model organisms when samples were small or degraded. I also created the necessary software to facilitate the analysis of organelle genomes, including new indices that check for the completeness of circular genomes or misalignments in other types of contigs and scaffolds.
}
%
\cventry{2018--2019}
{Professor Convidado}
{Universidade de São Paulo}
{São Paulo--SP}
{Brasil}
{
\textbullet~I created and taught the first graduate-level courses in bioinformatics and computer programming for biologists at USP's Graduate Program in Zoology, which is ranked first in animal research worldwide.
}
%
\cventry{Jul--Ago, 2017}
{Professor Convidado}
{Universidade de Magdalena}
{Santa Marta D.T.C.H.}
{Colômbia}
{
\textbullet~I designed and taught the first course in bioinformatics at the University of Magdalena. Classes included practical and theoretical lessons on DNA extraction and isolation, high-throughput sequencing, sequence alignment, reference, and \textit{de novo} genome assembly, and gene prediction and annotation.\\
\textbullet~I also offered consultation to faculty members involved in animal research who needed web and dry lab solutions for next-generation sequence analyses.
}

%%%%%%%%%%%%%%%%%%%%%%
%% FORMAL EDUCATION %%
%%%%%%%%%%%%%%%%%%%%%%

\vspace{-1em}

\section{Educação}

\cventry{2013--2018}
{Ph.D. em Bioinformática}
{Universidade de São Paulo}
{São Paulo--SP}
{Brasil}
{}

\cventry{2010--2012}
{M.Sc. em Zoologia}
{Universidade de São Paulo}
{São Paulo--SP}
{Brasil}
{}

\cventry{2006--2009}
{B.Sc. em Ciências Biológicas (habilitação em Biologia Marinha)}
{Universidade Estadual Paulista ``Júlio de Mesquita Filho''}
{São Vicente--SP}
{Brasil}
{}

O meu \textit{curriculum vitae} e outras informações profissionais estão disponíveis em \url{https://machadodj.github.io}.

\end{document}
